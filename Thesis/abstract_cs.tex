Kvalitní data jsou zásadní pro důvěryhodná rozhodnutí na datech založená. Značná část současných přístupů k měření kvality dat je spojena s náročnou, odbornou a časově náročnou prací, která vyžaduje manuální přístup k dosažení odpovídajících výsledků. Tyto přístupy jsou navíc náchylné k chybám a nevyužívají plně potenciál umělé inteligence (AI). Možným řešením je prozkoumat inovativní nové metody založené na strojovém učení (ML), které využívají potenciál AI k překonání těchto problémů.\\
\hspace*{5mm} Významná část práce se zabývá teorií kvality dat, která poskytuje komplexní vhled do této oblasti. V existující literatuře byly objeveny čtyři moderní metody založené na ML a byla navržena jedna nová metoda založená na autoenkodéru (AE).\\
\hspace*{5mm} Byly provedeny experimenty s AE a dolováním asociačních pravidel za pomoci metod zpracování přirozeného jazyka. Navrhované metody založené na AE prokázaly schopnost detekce potenciálních problémů s kvalitou dat na datasetech z reálného světa. Dolování asociačních pravidel dokázalo extrahovat byznys pravidla pro stanovený problém, ale vyžadovalo značné úsilí s předzpracováním dat. Alternativní metody nezaložené na AI byly také podrobeny analýze, ale vyžadovaly odborné znalosti daného problému a domény.
