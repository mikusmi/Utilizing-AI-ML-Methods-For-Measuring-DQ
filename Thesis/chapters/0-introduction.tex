\label{Introduction}

	Managing data quality is often associated with expensive and expert work that includes manual effort to achieve adequate results. These financial and human resources costs are usually a substantial factor in why managing data quality is challenging and is perceived only as an auxiliary service by the companies.
	
	In the worst case, the company purchases an expensive proprietary vendor lock-prone data quality tool, dedicates and trains several of its employees to learn how to use it. If the potential of these data quality tools is not fully realised the resulting business value is low.
	
	To avoid not only the above-mentioned shortcomings in managing data quality but also to be able to look for an innovative AI approach which would reduce expert or user intervention into the field of data quality measurement, the key areas related to data quality were analyzed in this thesis -- the nature of data \seesection{sec:understanding_data}, essentials of data quality \seesection{sec:understanding_data_quality} and the current data quality tools \seesection{sec:existing_data_quality_tools_review}.

	The motivation of this work is to explore and experiment with state-of-the-art approaches for measuring data quality with special emphasis on AI/ML-based methods that have the potential to reduce the amount of manual work required.
	
	This thesis uses the term data quality measurement in the context of various data quality activities (e.g. de-duplication, outliers detection, rules mining or monitoring as ongoing measurement) that are understood as an auxiliary part of the measurement. 
	
	In recent years, there has been a significant shift in the development of AI/ML-based methods, where their utilization helps to innovate across the industry, especially by shielding from human error and implementation of automation. Current general-purposed data quality tools do not take full advantage of the possibilities offered by these AI approaches. The reason for not using them on a broader scale may be the complexity of the application and utilization of these methods in the field of data quality.  
	
	For these reasons, the work analyzes potential of state-of-the-art approaches for measuring data quality with a focus on AI/ML \seesection{ch:data_quality_measurement_methods}. For the experimental part of the work \seesection{ch:experiments_and_results}, two AI/ML-based approaches to measuring data quality were selected -- Autoencoders \seesection{subsec:experiment_1_autoencoder} and Association Rule Mining using NLP \seesection{subsec:experiment_2_association_rule_mining}. Each of the approaches was compared with the nearest discovered complementary non-AI approach.